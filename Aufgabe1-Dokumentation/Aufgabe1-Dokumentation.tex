\documentclass[a4paper, notitlepage, 12pt]{scrartcl}
\author{Lukas Rost \\ \small{Teilnahme-ID: 48125}}
\title{Aufgabe 1 \\ \glqq Lisa rennt\grqq  - Dokumentation}
\subtitle{37. Bundeswettbewerb Informatik 2018/19 - 2. Runde \\~\\}
\date{29. April 2018}
\usepackage[ngerman]{babel}
\usepackage[utf8]{inputenc}
\usepackage{graphicx}
\usepackage{wrapfig}
\usepackage{color}
\usepackage[dvipsnames]{xcolor}
\usepackage{hyperref}
\usepackage[top=2.5cm, bottom=1.5cm, left=2.5cm, right=2.5cm]{geometry}
\usepackage{fancyvrb}
\usepackage{caption}
\usepackage{mathtools}
\usepackage{amssymb}
\usepackage{fancyhdr}
\usepackage{lastpage}

\usepackage{minted}
\fvset{breaklines=true}

\pagestyle{fancy}
\lhead{Lukas Rost, Teilnahme-ID: 48125}
\rhead{Aufgabe 1, Seite \thepage ~von \pageref{LastPage}}
\cfoot{ }

\newenvironment{longlisting}{\captionsetup{type=listing}}{}

\newmintedfile{java}{frame=single,linenos,samepage=false,firstnumber=1,rulecolor=\color{Gray},autogobble,breakafter=.u,fontsize=\small}

\begin{document}
\renewcommand{\contentsname}{\centerline{Inhaltsverzeichnis}}
 \maketitle
 \tableofcontents
 \thispagestyle{empty}
 \newpage
 \setcounter{page}{1}
 
 \section{Lösungsidee}
 \subsection{Das Geometric-Shortest-Path-Problem}
 \subsection{Erzeugung eines Sichtbarkeitsgraphen}
 \subsection{Lösung des Problems ohne Hindernisse}
 \subsection{Kombination der Ansätze}
\begin{thebibliography}{xx}
\bibitem[1] {Src:qwert} Wikipedia-Artikel zur NP-Vollständigkeit, \url{https://de.wikipedia.org/wiki/NP-Vollst\%C3\%A4ndigkeit}
\end{thebibliography}

\section{Umsetzung}
\subsection{Allgemeine Hinweise zur Benutzung}

\section{Beispiele}

 \section{Quellcode}
 \renewcommand{\listingscaption}{Quellcode}
 
 \end{document}