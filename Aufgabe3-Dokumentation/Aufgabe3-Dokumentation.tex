\documentclass[a4paper, notitlepage, 12pt]{scrartcl}
\author{Lukas Rost \\ \small{Teilnahme-ID: 48125}}
\title{Aufgabe 3 \\ \glqq Schach dem Wildschwein\grqq  - Dokumentation}
\subtitle{37. Bundeswettbewerb Informatik 2018/19 - 2. Runde \\~\\}
\date{29. April 2019}
\usepackage[ngerman]{babel}
\usepackage[utf8]{inputenc}
\usepackage{graphicx}
\usepackage{wrapfig}
\usepackage{color}
\usepackage[dvipsnames]{xcolor}
\usepackage{hyperref}
\usepackage[top=2.5cm, bottom=1.5cm, left=2.5cm, right=2.5cm]{geometry}
\usepackage{fancyvrb}
\usepackage{caption}
\usepackage{mathtools}
\usepackage{amssymb}
\usepackage{fancyhdr}
\usepackage{lastpage}

\usepackage{minted}
\fvset{breaklines=true}

\pagestyle{fancy}
\lhead{Lukas Rost, Teilnahme-ID: 48125}
\rhead{Aufgabe 3, Seite \thepage ~von \pageref{LastPage}}
\cfoot{ }

\newenvironment{longlisting}{\captionsetup{type=listing}}{}

\newmintedfile{java}{frame=single,linenos,samepage=false,firstnumber=1,rulecolor=\color{Gray},autogobble,breakafter=.u,fontsize=\small}

\begin{document}
\renewcommand{\contentsname}{\centerline{Inhaltsverzeichnis}}
 \maketitle
 \tableofcontents
 \thispagestyle{empty}
 \newpage
 \setcounter{page}{1}
 
 \section{Lösungsidee}
 \subsection{...?}
 \subsection{Das Drei-Springer-Endspiel}
 \subsection{...}
 \subsection{Laufzeitbetrachtung}
\begin{thebibliography}{xx}
\bibitem[1] {Src:BCE} Fine, Reuben: Basic Chess Endings, McKay-Verlag, 1941, ISBN 978-0-8129-3493-9
\end{thebibliography}

\section{Umsetzung}
\subsection{Allgemeine Hinweise zur Benutzung}
\subsection{Struktur des Programms}
\subsection{Implementierung der wichtigsten Algorithmen}

\section{Beispiele}
\subsection{Beispiel 1}
\subsection{Beispiel 2}
\subsection{Beispiel 3}
\subsection{Beispiel 4}
\subsection{Beispiel 5}
\subsection{Eigene Beispiele}

 \section{Quellcode}
 \renewcommand{\listingscaption}{Quellcode}
 
 \end{document}